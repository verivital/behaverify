%\documentclass[twocolumn]{acmart}
\documentclass[twocolumn]{article}
%\documentclass[]{article}
\usepackage{amsmath}
\usepackage{amssymb}
\usepackage{mathtools}
\usepackage[numbers]{natbib}
\usepackage{url}
%\usepackage{tikz}
%\usetikzlibrary{arrows,automata,fit,positioning}
%\usepackage[latin1]{inputenc}
%\usepackage[linewidth=1pt]{mdframed}
%\usepackage[margin=1.0in]{geometry}
%\usepackage{multicol}
%\usepackage{blindtext}

%\usepackage{xpatch}

%\makeatletter
%\xpatchcmd{\ps@firstpagestyle}{Manuscript submitted to ACM}{}{\typeout{First patch succeeded}}{\typeout{first patch failed}}
%\xpatchcmd{\ps@standardpagestyle}{Manuscript submitted to ACM}{}{\typeout{Second patch succeeded}}{\typeout{Second patch failed}}    \@ACM@manuscriptfalse% Also in titlepage
%\makeatother

%\settopmatter{printacmref=false} % Removes citation information below abstract
%\renewcommand\footnotetextcopyrightpermission[1]{} % removes footnote with conference info
%\setcopyright{none}
%\pagestyle{plain} %



%\bibliographystyle{ACM-Reference-Format}
\bibliographystyle{unsrtnat}
%\bibliographystyle{ksfh_nat}

% Title Page
\title{BehaVerify: A Tool for Verifying Behavior Trees with NUXMV}
\author{Bernard Serbinowski}

%%
%% \BibTeX command to typeset BibTeX logo in the docs
%\AtBeginDocument{%
%  \providecommand\BibTeX{{%
%    \normalfont B\kern-0.5em{\scshape i\kern-0.25em b}\kern-0.8em\TeX}}}



\begin{document}
%\begin{multicols]{2}


%%
%% The code below is generated by the tool at http://dl.acm.org/ccs.cfm.
%% Please copy and paste the code instead of the example below.
%%
%\begin{CCSXML}
%\end{CCSXML}

%\ccsdesc[500]{Computer systems organization~Embedded systems}
%\ccsdesc[300]{Computer systems organization~Redundancy}
%\ccsdesc{Computer systems organization~Robotics}
%\ccsdesc[100]{Networks~Network reliability}


\maketitle
\thispagestyle{empty}

\section*{Abstract}

Behavior Trees are a framework for describing the execution of a task. This includes sequences, which allow actions to be ordered, selectors, which allow priorities to be set, and other control flow options. While they originated in video games, they have since gained traction within the robotics community. Behavior Trees are split into composite, or internal, nodes and leaf, or external, nodes. Composite nodes control general tree traversal while leaf nodes consist of guard conditions and actions for the tree to execute. While the leaf nodes are frequently customized, the composite nodes are generally used out of the box. BehaVerify is a tool that leverages this to automatically create a model which only requires additional user customization for specialized leaf nodes. The model can be verified using nuXmv.

%\end{multicols}


%%
%% The acknowledgments section is defined using the "acks" environment
%% (and NOT an unnumbered section). This ensures the proper
%% identification of the section in the article metadata, and the
%% consistent spelling of the heading.
%\begin{acks}
%\end{acks}

%%
%% The next two lines define the bibliography style to be used, and
%% the bibliography file.
\bibliography{prelim}

%\blindtext[20]
\end{document}
