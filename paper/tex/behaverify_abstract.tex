\documentclass[conference]{IEEEtran}
\IEEEoverridecommandlockouts
% The preceding line is only needed to identify funding in the first footnote. If that is unneeded, please comment it out.
\usepackage{cite}
\usepackage{amsmath,amssymb,amsfonts}
\usepackage{algorithmic}
\usepackage{graphicx}
\usepackage{textcomp}
\usepackage{xcolor}
\def\BibTeX{{\rm B\kern-.05em{\sc i\kern-.025em b}\kern-.08em
    T\kern-.1667em\lower.7ex\hbox{E}\kern-.125emX}}
\begin{document}

\title{BehaVerify: A Tool for Verifying Behavior Trees with NUXMV\\
\thanks{DARPA}
}

\author{\IEEEauthorblockN{1\textsuperscript{st} Bernard Serbinowski}
\IEEEauthorblockA{\textit{dept. name of organization (of Aff.)} \\
\textit{name of organization (of Aff.)}\\
Nashville, TN, United States \\
bernard.serbinowski@vanderbilt.edu}
}

\maketitle

\begin{abstract}
Behavior Trees, which originated in video games as a method for controlling NPCs but have since gained traction within the robotics community, are a framework for describing the execution of a task. BehaVerify is a tool that creates a nuXmv model when given a py\_tree. For composite nodes this process is automatic and requires no additional user input, as these nodes are provided out of the box by py\_trees. A wide variety of leaf nodes are automatically supported and require no additional user input, but customized leaf nodes will require additional user input to be correctly modeled. BehaVerify can provide a template to make this easier. BehaVerify was able to create a nuXmv model with over 100 nodes and nuXmv was able to verify various non-trivial LTL and CTL properties on this model, both directly and via counterexample. The model in question features parallel nodes, selector, and sequence nodes.
\end{abstract}

\end{document}
